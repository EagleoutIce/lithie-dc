\documentclass{lecture-digital}

\usepackage{random-intropoem}
\RipWidth{0.7\textwidth}
\SetRipStyle{%
\bgroup\null\hfill\large\RipParBox{%
        \raggedleft\riptext%
        \smallskip\par%
    \itshape\ripname,\quad\ripdate\\%
    \ripauthor}\vspace*{2\baselineskip}\egroup}

\usepackage{blindtext}

\titleimage{\includegraphics[width=0.6\linewidth]{exampletitleimage}}

\title{Die Kunst der alten Schule}
\subtitle{Klaus Rüdiger}
\addAuthor{Florian Sihler (florian.sihler@uni-ulm.de)}
\brief{\nohyper Eine Mitschrift von: \typesetAuthor\\{\normalsize\timestamp~$\circ$~\mail{florian.sihler@uni-ulm.de}}}
\emblem[university ulm]{Praktische Informatik}{\faCode}

\begin{document}

\frontmatter
\maketitle

\thispagestyle{empty}
\makeatletter\lecture@coverpage@geometry%
\onecolumn\null\vfill%
\GetRandomRip
\restoregeometry{}
\clearpage
\tableofcontents
\mainmatter

\chapter{Superlangerkapitelname der gewiss nicht passen wird, denn das ist der Plan!}
\csummary{Wir beschäftigen uns mit Käse, insbesondere mit gutem, gereiften Käse! 
          Das finde ich so unglaublich subber-glasse, das ist eine zweite Zeile darüber wert, was wir in diesem Kapitel behandeln werden!
          Es wird bestimmt auch Fondue erwähnt.}

\section{Mist, er hat doch draufgepasst, das ist jetzt aber doof auch, da muss ich ihn noch länger machen}

\Blindtext[9]
\clearpage
\section{Alabambi}
Ich brauche eine Fußnote\footnote{Kurze Fußnote.}.
\Blindtext[6]
Hier wäre auch noch eine lange\footnote{\blindtext[1]}.

\Blinddocument
\foreach\i in {1,...,5}{
    \Blinddocument
}

\end{document}